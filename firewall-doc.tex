%% Dr Alun Moon
%% assignment-template.tex, generated from assignment-template.dtx
\documentclass[12pt]{article}
\usepackage[british]{babel}

\usepackage{url}
\usepackage[round]{natbib}
\usepackage{infoboxes}

\usepackage{latexsym}
\usepackage{tlatex}
\usepackage{color}
\definecolor{boxshade}{gray}{.8}
\setboolean{shading}{true}

\usepackage{enumerate,
            siunitx}

\begin{document}
\begin{abstract}
	This is one of the more complex specifications,  the catch is that most of
	the complexity is in the model!
\end{abstract}

\section{Simple Firewall}
\subsection{The Specification} 
\begin{tla}
------------------------------ MODULE Firewall ------------------------------
EXTENDS Naturals
CONSTANT AcceptList,
         RejectList,
         Protocols,
         IPs,
         Ports

VARIABLE forwarded \* packets forwarded on through the firewall
----
Address == [ ip:IPs, port:Ports ] \* address has port and ip

(*
A Packet has a source and destination address,
and a protocol
*)
Packet  == [ source:Address,
             dest:Address,
             protocol: Protocols ]

Rule == [ Packet -> BOOLEAN ] \* A Rule maps from packets to boolean values
----
Init == forwarded = {}

Reject(p) ==
    \E rule \in RejectList :
        /\ rule[p]

Accept(p) ==
    \E rule \in AcceptList :
        /\ rule[p]

Firewall(p) == \* Firewall acts on a packet
    IF Accept(p)/\\lnot Reject(p) 
    THEN forwarded' = forwarded \union {p}
    ELSE UNCHANGED forwarded
    
Next ==
    \E packet \in Packet : Firewall(packet)

=============================================================================
\* Modification History
\* Last modified Tue Feb 16 16:16:29 GMT 2021 by alunm
\* Created Sat Feb 13 18:27:34 GMT 2021 by alunm
\end{tla}
\begin{tlatex}
\@x{}\moduleLeftDash\@xx{ {\MODULE} Firewall}\moduleRightDash\@xx{}%
\@x{ {\EXTENDS} Naturals}%
\@x{ {\CONSTANT} AcceptList ,\,}%
\@x{\@s{60.21} RejectList ,\,}%
\@x{\@s{60.21} Protocols ,\,}%
\@x{\@s{60.21} IPs ,\,}%
\@x{\@s{60.21} Ports}%
\@pvspace{8.0pt}%
\@x{ {\VARIABLE} forwarded}%
\@y{%
  packets forwarded on through the firewall
}%
\@xx{}%
\@x{}\midbar\@xx{}%
\@x{ Address \.{\defeq} [ ip \.{:} IPs ,\, port \.{:} Ports ]}%
\@y{%
  address has port and ip
}%
\@xx{}%
\@pvspace{8.0pt}%
\begin{lcom}{0}%
\begin{cpar}{0}{F}{F}{0}{0}{}%
  
 A Packet has a source and destination address,
 and a protocol
 
\end{cpar}%
\end{lcom}%
\@x{ Packet\@s{4.1} \.{\defeq} [ source \.{:} Address ,\,}%
\@x{\@s{65.03} dest \.{:} Address ,\,}%
\@x{\@s{65.03} protocol \.{:} Protocols ]}%
\@pvspace{8.0pt}%
\@x{ Rule \.{\defeq} [ Packet \.{\rightarrow} {\BOOLEAN} ]}%
\@y{%
  A Rule maps from packets to boolean values
}%
\@xx{}%
\@x{}\midbar\@xx{}%
\@x{ Init \.{\defeq} forwarded \.{=} \{ \}}%
\@pvspace{8.0pt}%
\@x{ Reject ( p ) \.{\defeq}}%
\@x{\@s{16.4} \E\, rule \.{\in} RejectList \.{:}}%
\@x{\@s{29.16} \.{\land} rule [ p ]}%
\@pvspace{8.0pt}%
\@x{ Accept ( p ) \.{\defeq}}%
\@x{\@s{16.4} \E\, rule \.{\in} AcceptList \.{:}}%
\@x{\@s{29.16} \.{\land} rule [ p ]}%
\@pvspace{8.0pt}%
\@x{ Firewall ( p ) \.{\defeq}}%
\@y{%
  Firewall acts on a packet
}%
\@xx{}%
\@x{\@s{16.4} {\IF} Accept ( p ) \.{\land} {\lnot} Reject ( p )}%
\@x{\@s{16.4} \.{\THEN} forwarded \.{'} \.{=} forwarded \.{\cup} \{ p \}}%
\@x{\@s{16.4} \.{\ELSE} {\UNCHANGED} forwarded}%
\@pvspace{8.0pt}%
\@x{ Next \.{\defeq}}%
\@x{\@s{16.4} \E\, packet \.{\in} Packet \.{:} Firewall ( packet )}%
\@pvspace{8.0pt}%
\@x{}\bottombar\@xx{}%
\@x{}%
\@y{%
  Modification History
}%
\@xx{}%
\@x{}%
\@y{%
  Last modified Tue Feb 16 16:16:29 GMT 2021 by alunm
}%
\@xx{}%
\@x{}%
\@y{%
  Created Sat Feb 13 18:27:34 GMT 2021 by alunm
}%
\@xx{}%
\end{tlatex}

\section{The Model}


\subsection{Model Overview}
\paragraph{The Behaviour specification} is an \emph{Initial predicate and
next-state relation}
\begin{tla}
	Init
	Next
\end{tla}
\begin{tlatex}
\@x{\@s{32.8} Init}%
\@x{\@s{32.8} Next}%
\end{tlatex}

\paragraph{The Model} Is quite complex due to the nature of the problem and
the data required.
\subparagraph{Declared constants} the constants in the specification are
declared as follows.
\begin{description}
\item[Accept list] A list of functions that match packets
\begin{tla}
AcceptList <- {
	[p \in Packet |-> p.protocol = "http"],
	[p \in Packet |-> p.dest.port = 4],
	[p \in Packet |-> p.source.ip \in 1..3]
}
\end{tla}
\begin{tlatex}
\@x{ AcceptList \.{\leftarrow} \{}%
 \@x{\@s{32.8} [ p \.{\in} Packet \.{\mapsto} p . protocol \.{=}\@w{http} ]
 ,\,}%
 \@x{\@s{32.8} [ p \.{\in} Packet \.{\mapsto} p . dest . port\@s{0.24} \.{=} 4
 ] ,\,}%
 \@x{\@s{32.8} [ p \.{\in} Packet \.{\mapsto} p . source . ip \.{\in} 1
 \.{\dotdot} 3 ]}%
\@x{ \}}%
\end{tlatex}
\item[RejectList] is the list of rules to match rejected packets.
\begin{tla}
RejectList <- {}
\end{tla}
\begin{tlatex}
\@x{ RejectList \.{\leftarrow} \{ \}}%
\end{tlatex}
\item[Protocols] A set of strings of protocol names to match.
\begin{tla}
Protocols <- {"http","smtp","ftp"}
\end{tla}
\begin{tlatex}
\@x{ Protocols \.{\leftarrow} \{\@w{http} ,\,\@w{smtp} ,\,\@w{ftp} \}}%
\end{tlatex}
\item[Ports] The range of ports in use
\begin{tla}
Ports <- 1..5
\end{tla}
\begin{tlatex}
\@x{ Ports \.{\leftarrow} 1 \.{\dotdot} 5}%
\end{tlatex}

\item[IPs] The range if IP addresses to use
\begin{tla}
IPs <- 1..10
\end{tla}
\begin{tlatex}
\@x{ IPs \.{\leftarrow} 1 \.{\dotdot} 10}%
\end{tlatex}

\end{description}

\subsection{Checks and verifications}
The invariant is checked to see if all the packets that have been forwarded
are either accepted or not expicitly rejected.
\begin{tla}
	\A p \in forwarded : Accept(p) /\ \lnot Reject(p)
\end{tla}
\begin{tlatex}
 \@x{\@s{32.8} \A\, p \.{\in} forwarded \.{:} Accept ( p ) \.{\land} {\lnot}
 Reject ( p )}%
\end{tlatex}

\subsection{Results} A summary of the results
\paragraph{Statistics} a summaries of the actions and number of states
found.

\begin{table}[h]
\begin{tabular}{lr}
	States found & \num{7501} \\
 Distinct states & \num{1} \\ 
\end{tabular}
\end{table}

\begin{table}[h]
\begin{tabular}{llrr}
	\textbf{Action} & Location & States Found & \textbf{Distinct states} \\
	\hline
	\textit{Init}   & Line 23 & 1 & 1 \\
	\textit{Firewall}   & Line 33 & \num{7500} & 0 \\
\end{tabular}
\end{table}

\subsection{Discussion}
\subsubsection{Model description} 
This is a complex model.  In this case we have to model and specify the
environment in which the specification operates.

\paragraph{The state of the system is\ldots}
The state of the firewall is just the contents of the ACL, packets coming
into the firewall are not really part of the firewall state, but are part of
the environment state in which the firewall operates.
The \emph{forwarded} and \emph{declined} variables model this state of the
environment.

\paragraph{Rules}  The firewall uses rules to check packets and accept or
reject them if they match.  A rule can be thought of as a predicate function
on the packets, i.e. it is a function that maps from packets to a boolean
value.

It's type is therefore:
\begin{tla}
	Rule == [ Packet -> BOOLEAN ]
\end{tla}
\begin{tlatex}
\@x{\@s{32.8} Rule \.{\defeq} [ Packet \.{\rightarrow} {\BOOLEAN} ]}%
\end{tlatex}

The accept-list and reject-list are simply sets of rules.

\paragraph{ACL} The access control list is a set of accepting and rejecting
rules.  These are constants from the model.  This could be a variable and
managed by operations, but it adds nothing to the firewall behaviour. 

\paragraph{Accept and Reject rules} A packet is accepted (or rejected) if
there is a rule that matches the packet, in other words
\begin{tla}
	\E rule \in AcceptList : rule[p]
\end{tla}
\begin{tlatex}
\@x{\@s{32.8} \E\, rule \.{\in} AcceptList \.{:} rule [ p ]}%
\end{tlatex}
This is true, if there is a rule that is true, and a rule is true if it
matches a packet.
A rule might be:
\begin{tla}
	[ p \in Packet |-> p.dest.port = 8080 ]
\end{tla}
\begin{tlatex}
\@x{\@s{32.8} [ p \.{\in} Packet \.{\mapsto} p . dest . port \.{=} 8080 ]}%
\end{tlatex}
Which is true if the port field of the dest field in the packet record is
8080, and false otherwise.

\paragraph{Firewall Action}
The action of the fire wall is to forward a packet if it is in the accept list
and not matched by a reject rule.
\begin{tla}
Firewall(p) == \* Firewall acts on a packet
    IF Accept(p)/\\lnot Reject(p)
    THEN forwarded' = forwarded \union {p}
    ELSE UNCHANGED forwarded
\end{tla}
\begin{tlatex}
\@x{ Firewall ( p ) \.{\defeq}}%
\@y{%
  Firewall acts on a packet
}%
\@xx{}%
\@x{\@s{16.4} {\IF} Accept ( p ) \.{\land} {\lnot} Reject ( p )}%
\@x{\@s{16.4} \.{\THEN} forwarded \.{'} \.{=} forwarded \.{\cup} \{ p \}}%
\@x{\@s{16.4} \.{\ELSE} {\UNCHANGED} forwarded}%
\end{tlatex}

\paragraph{Next action}
The next action is that there is a packet that can be forwarded.
\begin{tla}
Next ==
    \E packet \in Packet : Firewall(packet)
\end{tla}
\begin{tlatex}
\@x{ Next \.{\defeq}}%
\@x{\@s{16.4} \E\, packet \.{\in} Packet \.{:} Firewall ( packet )}%
\end{tlatex}


\subsubsection{Interpretation of results}
This is an interesting case in that the firewall itself has no state, all the
states come from the environment, packets received, forwarded and rejected.
It is another system where care is needed in determining the size of the model
and consequently the size of the state-space.

\end{document}

\endinput
%%
%% End of file `assignment-template.tex'.
